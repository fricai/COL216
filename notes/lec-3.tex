\documentclass{scrartcl}

\usepackage{amssymb, amsthm, amsmath}

\usepackage{minted}

\title{COL216 Lecture 3}
\author{Rishabh Dhiman}
\date{13 January 2022}

\usepackage{listings}

\begin{document}
\maketitle

\section{Format for DP Instructions}

\subsection{Shift Operations on Registers}
Shift type:
\begin{itemize}
	\item LSL -- logical shift left
	\item LSR -- logical shift right
	\item ASR -- arithmetic shift right
	\item ROR -- rotate right
\end{itemize}
Shift amount can be,
\begin{itemize}
	\item 5 bit unsigned constant for shift amount,
	\item 4 bit register.
\end{itemize}
Rm, LSL \#4 -- const type 0 Rm
Rm, LSL Rs -- Rs 0 type 1 Rm

\subsection{Rotate operation on Constant}
\begin{tabular}{cc}
	\hline
	rot & lmm\\
	\hline
	4 & 8
\end{tabular}

lmm is an 8 bit constant (0 to 255) which is zero extended to 32 bits, and rotated right by 2 $\times$ rot bits.

\begin{itemize}
	\item operand2 = \#400 $\to$ lmm = 100, rot = 15, corresponds to right shift by $32 - 2 \times 15$,
	\item operand2 = \#800 $\to$ lmm = 50, rot = 14, correspnods to right shift by $32 - 2 \times 14$.
\end{itemize}

\subsection{Multiply}
Format for mul r1, r2, r3
\begin{tabular}{|c|c|c|c|c|c|c|c|c|c|}
	\hline
	& F & l & opc & & Rn & Rd & Rs & 1001 & Rm\\
	\hline
	4 & 2 & 1 & 4 & 1 & 4 & 4 & 4 & 4 & 4\\
	\hline
\end{tabular}

\subsection{Multiply Accumulate}
Format for mla r1, r2, r3, r4; -- r1 $\leftarrow$ r2 $\times$ r3 + r4.

\begin{tabular}{|c|c|c|c|c|c|c|c|c|c|}
	\hline
	& 00 & 0 & opc & & Rn & Rd & Rs & 1001 & Rm\\
	\hline
\end{tabular}

\subsection{Multiplying by a constant}
mul r1, r2, \#10 $\to$ mov r3, \#10; mul r1, r2, r3;

Multiplying by a power of 2 is equivalent to a logical left shift.

\section{Format for DT Instructions}
Rd $\leftarrow$ Memory [Rn + offset]

\begin{tabular}{|c|c|c|c|c|c|c|}
	\hline
	& F & opc & & Rn & Rd & offset\\
	\hline
	4 & 2 & 6 & 1 & 4 & 4 & 12\\
	\hline
\end{tabular}

F = 01.

The 12-bit offset can be
\begin{itemize}
	\item 12 bit unsigned constant,
	\item 4 bit register number, 8 shift specification
\end{itemize}
\subsection{Load}
ldr r4, [r5, \#32], opc = 25

\subsection{Indexing an Array}
\begin{minted}{arm}
mul r4, r5, #4
add r2, r2, r4
ldr r6, [r2, #0]
\end{minted}
\begin{minted}{arm}
add r2, r2, r5, LSL #4
ldr r6, [r2]
\end{minted}
Further reduction (fill it in):
\begin{minted}{arm}
ldr r6, [r2, r5, LSL #4]
\end{minted}

\subsection{Opcode field in DT Instructions}
6 opcode bits specify,
\begin{itemize}
	\item I (immediate): constant or register with shift
	\item P (pre/post indexing): pre or post indexing
	\item U (up/down): whether to add or subtract offset
	\item B (byte): byte or word transfer
	\item W (write back): whether to write back address into base register (Rn) or not.
	\item L (load/store): load from memory or store from memory
\end{itemize}
Post-indexing only makes sense with write-back.

Load half-word is a completely unrelated operation.

\begin{itemize}
	\item \mintinline{arm}{ldr r4, [r5, -r6]}
	\item \mintinline{arm}{str r4, [r5, r6, LSL #2]}
	\item \mintinline{arm}{ldrb r4, [r5, #32]!}
	\item \mintinline{arm}{strb r4, [r5, #-32]}
	\item \mintinline{arm}{ldr r4, [r5], r6}
	\item \mintinline{arm}{ldr r4, [r5], r6, LSL #2}
\end{itemize}

\subsubsection{Using auto-incrememnt/decrement}
When pointer is at first location vs when it points to the (possibly empty) element before address location.
\begin{itemize}
	\item get/pop: post-increment, pre-increment
	\item put/push: pre-decrement, post-decrement
\end{itemize}
\end{document}
