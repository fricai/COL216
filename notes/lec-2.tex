\documentclass{scrartcl}

\usepackage{amssymb, amsthm, amsmath}

\title{COL216 Lecture 2}
\author{Rishabh Dhiman}
\date{10 January 2022}

\usepackage{listings}

\begin{document}
\maketitle

\section{Control Flow}
pc is a program counter

\begin{lstlisting}
func:
	mov pc, lr
\end{lstlisting}
\begin{lstlisting}
void func() {
	return;
}
\end{lstlisting}

\begin{lstlisting}
X: func();

Y: func();
\end{lstlisting}

\begin{lstlisting}
bl func

bl func
\end{lstlisting}

Special registers
\begin{enumerate}
	\item r15 -- pc -- program counter
	\item r14 -- lr -- link register
	\item r13 -- sp -- stack pointer
\end{enumerate}

How to have nested or recursive calls?

\subsection{Passing Parameters}
\begin{enumerate}
	\item via registers
		\begin{lstlisting}
caller:
	move parameters into registers
	bl callee
	take result from register
		\end{lstlisting}
		\begin{lstlisting}
callee:
	access parameters in registers

	mov pc, lr
		\end{lstlisting}
	\item via stack
		\begin{lstlisting}
caller:
	move parameters into stack
	bl callee
	pop result from stack
		\end{lstlisting}
		\begin{lstlisting}
callee:
	access parameters in stack

	mov pc, lr
		\end{lstlisting}
\end{enumerate}
\subsubsection{Convention}
\begin{itemize}
	\item First 4 parameters in r0, r1, r2, r3.
	\item Result in r0.
	\item Beyond this, use stack.
	\item Callee can destroy r0, r1, r2, r3, r12.
	\item It should preserve other registers, except pc.
	\item Caller should preserver r0, r1, r2, r3, r12.
\end{itemize}

So the ``fixed'' code is
\begin{lstlisting}
caller:
	save registers
	move parameters into registers
	bl callee
	take result from register
	restore registers
\end{lstlisting}
\begin{lstlisting}
callee:
	save registers
	access parameters in registers
	restore registers
	mov pc, lr
\end{lstlisting}

Example:
\begin{lstlisting}
	g = A[0];
	for (i = 1; i < n; ++i)
		g = gcd(g, a[i]);
\end{lstlisting}

Sum array program:
\begin{lstlisting}
.equ SWI_Exit 0x11
.text
mov r1, #0 // load first element here instead
ldr r3, =AA // address location
add r6, r3, #400
L: ldr r5, [r3, #0]
add r1, r1, r5 // want to replace this with r1 = gcd(r1, r5)
add r3, r3, #4
cmp r3, r6 @ r6 = q
blt L
swi SWI_Exit
.data
AA: .space 400
.end
\end{lstlisting}
gcd:
\begin{lstlisting}
ldr r4, =AA // start address location
add r6, r4, #400 // end location
mov r0, [r4, #0]
add r4, r4, #4
L: ldr r1, [r4, #0]
bl gcd
add r4, r4, #4
cmp r4, r6 @ r6 = q
blt L
\end{lstlisting}
Note, we replace r0, r1, r2, r3 here with other things since they may not be preserved by gcd call.
\begin{lstlisting}
gcd: cmp r0, r1
	beq ret
	blt sub10
sub01: sub r0, r0, r1
	b gcd
sub10: sub r1, r1, r0
	b gcd
ret: mov pc, lr
\end{lstlisting}

\section{DP (data processing) instructions}
\begin{itemize}
	\item Arithmetic -- 
\end{itemize}
Fill this in

In an arithmetic operation, the second argument could be a constant too.
\subsection{Arithmetic}
\begin{itemize}
	\item add
	\item  sub
	\item rsb -- reverse subtract
	\item adc
	\item sbc
	\item rsc
\end{itemize}
c stands for with carry

\subsection{Logical}
\begin{itemize}
	\item and -- bit by bit logical AND
	\item orr -- bit by bit logical OR
	\item eor -- bit by bit logical XOR
	\item bic -- bit clear, op1 and not op2
\end{itemize}

\subsection{Move}
\begin{itemize}
	\item mov -- dest $\leftarrow$ src
	\item mvn -- dest $\leftarrow$ not src
\end{itemize}

\subsection{Compare}
\begin{itemize}
	\item cmp -- op1 - op2
	\item cmn -- op1 - (-op2)
\end{itemize}

\subsection{Test}
Modifies status flags.
\begin{itemize}
	\item tst -- op1 and op2
	\item teq -- op1 eor op2
\end{itemize}

\section{DT (Data Transfer) Instructions}
\begin{itemize}
	\item ldr / str -- load / store word
	\item ldrb / strb -- load / store byte
	\item ldrh / strh -- load / store half word
	\item ldrsb / ldrsh -- load signed byte / half word
	\item ldm / stm -- load / store multiple (any subset of registers can be specified)
\end{itemize}

\section{Comparison in ARM}
Signed comparison
\begin{itemize}
	\item equal -- beq
	\item not equal -- bne
	\item greater or equal -- bgeq
	\item less than -- blt
	\item greater than -- bgt
	\item less or equal -- ble
\end{itemize}

Unsigned comparison
\begin{itemize}
	\item equal -- beq
	\item not equal -- bne
	\item higher or same -- bhs
	\item lower -- blo
	\item higher -- bhi
	\item lower or same -- bls
\end{itemize}

\section{Condition Codes and Flags}
\begin{enumerate}
	\setcounter{enumi}{0}
	\item eq -- $Z = 1$
	\item ne -- $Z = 0$
	\item hs / cs (C set) $C = 1$ -- for unsigned values $(x - y)$ is implemented as $x + (2^n - y) = x + \tilde y$.
	\item lo / cc (C clear) $C = 0$
	\item mi (minus) $N = 1$
	\item pl (plus) $N = 0$
	\item vs (V set) $V = 1$
	\item vc (V clear) $V = 0$
	\item hi $C = 1$ and $Z = 0$
	\item ls $C = 0$ or $Z = 1$
	\item ge $N = V$
	\item ge $N \neq V$
	\item gt $N = V$ and $Z = 0$
	\item le $N \neq V$ or $Z = 1$
	\item al ignore all flags -- b = bal
\end{enumerate}

\section{Status Flags}
\begin{itemize}
	\item N -- Negative
	\item Z -- Zero
	\item C -- Carry
	\item V -- Overflow
\end{itemize}

\section{Assembler Directives}
.text
.data
.end

\begin{itemize}
\item .space -- reserve some space
\item .word -- reserve + initialize space
\item .byte
\item .ascii
\item .asciz
\end{itemize}

.equ -- equates a symbol to a value

\section{I/O in Arm}
SWI Instruction -- Instruction to invoke some service provided by system software.

Use in ARMSim:
\begin{itemize}
	\item I/O from/to stdin/stdout
	\item input/output from/to files
	\item opening/closing of files
	\item Halt execution
\end{itemize}

Example in ARMSim\# 1.91
\begin{lstlisting}
ldr r0, =message
swi 0x02 @ write on stdout

message: .asciz "Welcome\n"
\end{lstlisting}

I/O Example in ARMSim\# 2.01
\begin{lstlisting}
ldr r1, =param
mov r4, #1 @ file #1 is stdout
str r4, [r1]
ldr r4, =message
str r4, [r1, #4]
mov r4, #8 @ number of bytes
str r4, [r1, #8]
mov r0, #5 @ code for write
swi 0x123456

param: .word 0, 0, 0
message: .ascii "Welcome\n"
\end{lstlisting}

\section{ISR vs User Mode}
\end{document}
