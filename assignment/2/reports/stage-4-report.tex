\documentclass[a4paper]{scrartcl}

\usepackage{amsmath, amssymb}

\usepackage[T1]{fontenc}
\usepackage[a4paper, total={6.5in, 8.7in}]{geometry}

\usepackage{multicol}

\usepackage{minted}

\usepackage{hyperref}
\usepackage{cleveref}

\usepackage{pdfpages}

\title{COL216 Assignment 2\\{\Large Stage 4}}
\date{6 March 2022}
\author{Rishabh Dhiman\\ 2020CS10837}

\renewcommand{\tt}{\mintinline{text}}
\newcommand{\reg}{\mintinline{arm}}

\newcommand{\addwaveformpdf}[1]{
	\includepdf[landscape=true, picturecommand*={\put (\LenToUnit{.05\paperwidth},20) {\tt{#1.pdf}};}]{../output/#1.pdf}
}
\newcommand{\addprogramcode}[1]{%
\begin{minipage}{.25\textwidth}%
    \inputminted[linenos]{text}{../code/arm/binary/#1.txt}%
\end{minipage}%
\begin{minipage}{.7\textwidth}%
    \inputminted{arm}{../code/arm/assembly/#1.s}%
\end{minipage}%
}

\begin{document}
\maketitle

\section{Objective}
Test the processor designed in previous stages using ARM assembly codes.

\section{Technical Details}
\begin{itemize}
	\item The VHDL code was analyzed, and simulated using GHDL 1.0.0.
	\item The waveform viewer used is GTKWave Analyzer v3.3.104.
\end{itemize}

\section{Documentation}
The submission contains a single \tt{arm} directory along with a \tt{makefile}.

The \tt{arm} directory contains three folders, \tt{scripts}, \tt{assembly} and \tt{binary}.
\begin{itemize}
    \item \tt{scripts} directory contains two bash scripts \tt{to-bin.sh} and \tt{create-mem.sh}.
    \item \tt{assembly} directory contains three assembly codes, \tt{fact.s}, \tt{range-sum.s}, and \tt{neg-fib.s}.
    \item \tt{binary} directory contains three text files \tt{fact.txt}, \tt{range-sum.txt}, and \tt{neg-fib.txt}.
\end{itemize}

\section{Testing Procdure}
The code was analyzed, and simulated using GHDL. It was synthesized using Quartus 21.1.

You can simulate it yourself by copying the files to the folder containing the stage 3 submission.
\begin{enumerate}
    \item Ensure that you are using a Linux machine with Make and GHDL installed.
    \item Create a directory \tt{simulation/arm-tests/} exists (if it doesn't already exist).
    \item In the commandline, run \tt{make testarm INPUT=} $X$ where $X \in \{$\tt{fact}, \tt{range-sum}, \tt{neg-fib}$\}$.
    \item A $X$\tt{.ghw} waveform file will be created in the \tt{simulation/arm-tests} directory. Note that the earlier file will be overwritten.
\end{enumerate}

\section{Script Description}
The script, \tt{to-bin.sh} takes as argument the path to an ARM file and outputs the corresponding binary instruction code. It assumes that \tt{arm-linux-gnueabi} toolchain is installed, along with some common linux binaries.

The script, \tt{create-mem.sh} takes as argument the path to an ARM file and outputs the corresponding VHDL memory file with the ARM code hardcoded into it.

\section{Program Descriptions}
The description of the three programs follow,

\subsection{\texttt{fact}}

This computes $3!$ and stores the result in \reg{r0}. The code and the corresponding binary-encoded instructions are,

\addprogramcode{fact}
\clearpage

\subsection{\texttt{range-sum}}
This computes and stores $0 + 1 + \dots + i$ at memory location $64 + i$, for $0 \le i < 5$. This computation is done as $a[i] = a[i - 1] + i$, $a[0] = 0$. So that \reg{ldr} and \reg{str} are properly tested.

\addprogramcode{range-sum}
\clearpage

\subsection{\texttt{neg-fib}}
This computes the negative Fibonacci numbers, $F_{-i}$. If $a_i = F_{-i}$, then
\[F_{i + 2} = F_{i + 1} + F_{i} \implies F_i = F_{i + 2} - F_{i + 1} \implies a_i = F_{-i} = F_{-i + 2} - F_{-i + 1} = a_{i - 2} - a_{i - 1}.\]
$F_{-i}$ is stored in memory location, $64 + i$, for $0 \le i \le 8$.

\addprogramcode{neg-fib}
\clearpage

\section{Results}
All the testbenches gave the expected results.

The output waveform of the testbench results can be viewed in .ghw and .pdf files in the output folder.

The pdf files of the simulation are also attached at the end of this report.

\addwaveformpdf{arm-tests/fact}
\addwaveformpdf{arm-tests/range-sum}
\addwaveformpdf{arm-tests/neg-fib}
\end{document}
