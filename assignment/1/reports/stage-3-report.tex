\documentclass[a4paper]{scrartcl}

\usepackage[a4paper, total={6in, 8in}]{geometry}

\usepackage{minted}

\title{COL226 Assignment 1\\{\Large Stage 3}}
\date{3 February 2022}
\author{Rishabh Dhiman\\ \texttt{2020CS10837}}

\usepackage{amsthm, amsmath, amssymb}
\usepackage{hyperref}

\theoremstyle{definition}
\newtheorem{definition}{Definition}

\renewcommand{\tt}{\mintinline{text}}
\newcommand{\fun}{\mintinline{cpp}}
\newcommand{\reg}{\mintinline{text}}

\begin{document}

\maketitle

\section{Objective}
Construct a function to sort a list of strings using merge sort.

\section{Technical Details}
\begin{itemize}
	\item ARMSim version 2.0.1, Angel SWI Instructions
	\item Credits for \tt{io.s} to Ramanuj Goel, 2020CS510437 -- \href{https://github.com/taitaisama/Arm_io}{Github Link}
	\item Refer to stage 1 report for the definition of a case-insensitive comparison.
	\item In case duplicates are being removed, the string to be deleted is arbitrarily chosen.
	\item It's assumed that the total space taken by the input strings won't exceed 4096 bytes and that there will be atmost 1024 strings.
\end{itemize}

\section{Documentation}
All the files defining functions from Stage 1 and 2 of the assignment are used. Two new files, \tt{sort.s} and \tt{test_sort.s} have been added.

\subsection*{\tt{sort.s}}
This files defines two functions \fun{sort} and \fun{merge_sort} function with the C++ signature,
\begin{minted}{cpp}
int merge_sort(char** a, int len, int mode, char** res);
pair<char**, int> sort(char** a, int len, int mode);
\end{minted}

\subsubsection*{\mintinline{c}{merge_sort}}
It sorts the list of strings \tt{a}, and stores it in \tt{res}.
\begin{itemize}
	\item \tt{a} is an array of strings.
	\item \tt{len} is the length of \tt{a}.
	\item \tt{mode} is an integer parameters to control the way in which the lists are merged
		\begin{itemize}
			\item if the lowest bit of \tt{mode} is set, then case-insensitive comparison takes place,
			\item if the second bit of \tt{mode} is set, then equal strings are deleted while merging.
		\end{itemize}
	\item \tt{res} is a word-aliged memory location where the merged is stored.
	\item The function finally returns a single integer representing the length of the sorted list.
\end{itemize}

In ARM, the arguments \tt{a}, \tt{len}, \tt{mode}, \tt{res} correspond to the registers \reg{r0}, \reg{r1}, \reg{r2}, \reg{r3}, respectively. The output, length, is stored in \reg{r0}.

\subsubsection*{\mintinline{c}{sort}}
This functions is very similar to the previous \fun{merge_sort} function, however, rather than taking in the output array as \tt{res} an argument it returns a pointer to the sorted list along with its length.

In ARM, the arguments \tt{a}, \tt{len}, \tt{mode} correspond to the registers \reg{r0}, \reg{r1}, \reg{r2}, respectively. The output, sorted list and its length, correspond to \reg{r0} and \reg{r1}, respectively.

\subsection*{\tt{test_sort.s}}
This file is used to test the \fun{sort} function defined in \tt{sort.s}. The file interacts with the user via console,
\begin{itemize}
	\item It first asks the user, if the comparison done is case-insensitive or not.
	\item It then asks the user, if duplicate strings are to be removed or not.
	\item It then asks the user the for the number of strings, and then inputs the ASCII strings, each string in a new line.
\end{itemize}
It finally outputs the merged list of strings to the console, each string on a new line.

\section{Tests and Results}
To reproduce these, load \tt{io.s}, \tt{compare.s}, \tt{merge.s}, \tt{sort.s} and \tt{test_sort.s} in ARMSim. Run it and follow the instructions printed on the console.
\begin{enumerate}
	\item
\begin{minted}[breaklines]{text}
Case-insensitive compare (1 for case-insensitive comparison, 0 for case-sensitive): 0
Remove duplicates from list (1 to remove duplicates, 0 to not remove): 0
Number of strings in the list: 10
Input the strings, each string on a new line:
one
one
One
two
Three
four
five
six
seven
Eight

The sorted list is: 
Eight
One
Three
five
four
one
one
seven
six
two
\end{minted}
	\item
\begin{minted}[breaklines]{text}
Case-insensitive compare (1 for case-insensitive comparison, 0 for case-sensitive): 0
Remove duplicates from list (1 to remove duplicates, 0 to not remove): 1
Number of strings in the list: 10
Input the strings, each string on a new line:
one
two
One
three
four
one
five
six
seven
Eight

The sorted list is: 
Eight
One
five
four
one
seven
six
three
two
\end{minted}
	\item
\begin{minted}[breaklines]{text}
Case-insensitive compare (1 for case-insensitive comparison, 0 for case-sensitive): 1
Remove duplicates from list (1 to remove duplicates, 0 to not remove): 0
Number of strings in the list: 10
Input the strings, each string on a new line:
one
two
One
three
four
one
five
six
seven
Eight

The sorted list is: 
Eight
five
four
one
One
one
seven
six
three
two
\end{minted}
	\item
\begin{minted}[breaklines]{text}
Case-insensitive compare (1 for case-insensitive comparison, 0 for case-sensitive): 1
Remove duplicates from list (1 to remove duplicates, 0 to not remove): 1
Number of strings in the list: 10
Input the strings, each string on a new line:
one
two
One
three
four
one
Five
six
seven
eight

The sorted list is: 
eight
Five
four
one
seven
six
three
two
\end{minted}
\end{enumerate}
\end{document}
