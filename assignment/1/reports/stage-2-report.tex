\documentclass[a4paper]{scrartcl}

\usepackage[a4paper, total={6in, 8in}]{geometry}

\usepackage{minted}

\title{COL226 Assignment 1\\{\Large Stage 2}}
\date{28 January 2022}
\author{Rishabh Dhiman\\ \texttt{2020CS10837}}

\usepackage{amsthm, amsmath, amssymb}
\usepackage{hyperref}

\theoremstyle{definition}
\newtheorem{definition}{Definition}

\renewcommand{\tt}{\mintinline{text}}
\newcommand{\fun}{\mintinline{arm}}
\newcommand{\reg}{\mintinline{text}}

\begin{document}

\maketitle

\section{Objective}
Construct a function to merge two sorted list of strings.

\section{Technical Details}
\begin{itemize}
	\item ARMSim version 2.0.1, Angel SWI Instructions
	\item Credits for \tt{io.s} to Ramanuj Goel, 2020CS510437 -- \href{https://github.com/taitaisama/Arm_io}{Github Link}
	\item The input list of strings is sorted.
	\item If duplicates are being removed while merging, no two strings in the same list should be equal. The string to be deleted is arbitrarily chosen.
	\item Refer to stage 1 report for the definition of a case-insensitive comparison.
\end{itemize}

\section{Documentation}
All the files from Stage 1 of the assignment are used, along with this \tt{io.s} has been updated, two new files, \tt{merge.s} and \tt{test_merge.s} have been added.

\subsection*{\tt{merge.s}}
This files defines a merge function with the C signature,
\begin{minted}{c}
int merge(char** a, int n, char** b, int m, int mode, char** c);
\end{minted}
It merges the sorted list of strings \tt{a} and \tt{b}, and stores them in \tt{c}.
\begin{itemize}
	\item \tt{a} and \tt{b} are array of sorted strings.
	\item \tt{n} and \tt{m} are the length of \tt{a} and \tt{b}, respectively.
	\item \tt{mode} is an integer parameters to control the way in which the lists are merged
		\begin{itemize}
			\item if the lowest bit of \tt{mode} is set, then case-insensitive comparison takes place,
			\item if the second bit of \tt{mode} is set, then equal strings are deleted while merging.
		\end{itemize}
	\item \tt{c} is a word-aliged memory location where the merged list is stored.
	\item The function finally returns a single integer representing the length of the merged list.
\end{itemize}
The \fun{merge} function doesn't change any value in \tt{a} or \tt{b} and simply copies the string pointers rather than copying the string.

\subsection*{\tt{test_merge.s}}
This file is used to test the \fun{merge} function defined in \tt{merge.s}. The file interacts with the user via console,
\begin{itemize}
	\item It first asks the user, if the comparison done is case-insensitive or not.
	\item It then asks the user, if duplicate strings are to be removed or not.
	\item It then asks the length of the first sorted list, and then inputs the ASCII strings, each string in a new line.
	\item It then asks the length of the second sorted list, and then inputs the ASCII strings, each string in a new line.
\end{itemize}
It finally outputs the merged list of strings to the console, each string on a new line.

\section{Tests and Results}
To reproduce these, load \tt{io.s}, \tt{compare.s}, \tt{merge.s} and \tt{test_merge.s} in ARMSim. Run it and follow the instructions printed on the console.
\begin{enumerate}
	\item
\begin{minted}[breaklines]{text}
Do a case-insensitive comparison? (1 for case-insensitive comparison, 0 for case-sensitive): 0
Remove duplicates from list, in case dulicates are to be removed no equal elements should be present in the same list (1 to remove duplicates, 0 to not remove): 0
Number of strings in the first list: 3
Input the strings in a sorted order, each string on a new line:
one
three
two
Number of strings in the second list: 2
Input the strings in a sorted order, each string on a new line:
four
six
The merged list is: 
four
one
six
three
two
\end{minted}
	\item
\begin{minted}[breaklines]{text}
Do a case-insensitive comparison? (1 for case-insensitive comparison, 0 for case-sensitive): 1
Remove duplicates from list, in case dulicates are to be removed no equal elements should be present in the same list (1 to remove duplicates, 0 to not remove): 0
Number of strings in the first list: 3
Input the strings in a sorted order, each string on a new line:
One
three
two
Number of strings in the second list: 3
Input the strings in a sorted order, each string on a new line:
Four
one
two
The merged list is: 
Four
One
one
three
two
two
\end{minted}
	\item
\begin{minted}[breaklines]{text}
Do a case-insensitive comparison? (1 for case-insensitive comparison, 0 for case-sensitive): 0
Remove duplicates from list, in case dulicates are to be removed no equal elements should be present in the same list (1 to remove duplicates, 0 to not remove): 1
Number of strings in the first list: 3
Input the strings in a sorted order, each string on a new line:
one
three
two
Number of strings in the second list: 4
Input the strings in a sorted order, each string on a new line:
Two
one
six
three
The merged list is: 
Two
one
six
three
two
\end{minted}
	\item
\begin{minted}[breaklines]{text}
Do a case-insensitive comparison? (1 for case-insensitive comparison, 0 for case-sensitive): 1
Remove duplicates from list, in case dulicates are to be removed no equal elements should be present in the same list (1 to remove duplicates, 0 to not remove): 1
Number of strings in the first list: 3
Input the strings in a sorted order, each string on a new line:
one
three
two
Number of strings in the second list: 4
Input the strings in a sorted order, each string on a new line:
One
six
Three
two
The merged list is: 
one
six
three
two
\end{minted}
\end{enumerate}
\end{document}
